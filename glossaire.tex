% Glossaire et acronymes
% Ajoutez ce contenu dans votre préambule si vous souhaitez utiliser un glossaire

% Packages nécessaires (à ajouter dans le préambule)
% \usepackage[acronym,toc]{glossaries}
% \makeglossaries

% Définition des acronymes
\newacronym{api}{API}{Application Programming Interface}
\newacronym{rest}{REST}{Representational State Transfer}
\newacronym{crud}{CRUD}{Create, Read, Update, Delete}
\newacronym{sql}{SQL}{Structured Query Language}
\newacronym{nosql}{NoSQL}{Not Only SQL}
\newacronym{json}{JSON}{JavaScript Object Notation}
\newacronym{xml}{XML}{eXtensible Markup Language}
\newacronym{html}{HTML}{HyperText Markup Language}
\newacronym{css}{CSS}{Cascading Style Sheets}
\newacronym{js}{JS}{JavaScript}
\newacronym{ui}{UI}{User Interface}
\newacronym{ux}{UX}{User Experience}
\newacronym{mvc}{MVC}{Model-View-Controller}
\newacronym{orm}{ORM}{Object-Relational Mapping}
\newacronym{ci}{CI}{Continuous Integration}
\newacronym{cd}{CD}{Continuous Deployment}
\newacronym{devops}{DevOps}{Development Operations}
\newacronym{aws}{AWS}{Amazon Web Services}
\newacronym{gcp}{GCP}{Google Cloud Platform}
\newacronym{iot}{IoT}{Internet of Things}
\newacronym{ai}{AI}{Artificial Intelligence}
\newacronym{ml}{ML}{Machine Learning}
\newacronym{dl}{DL}{Deep Learning}
\newacronym{bi}{BI}{Business Intelligence}
\newacronym{etl}{ETL}{Extract, Transform, Load}
\newacronym{olap}{OLAP}{Online Analytical Processing}
\newacronym{oltp}{OLTP}{Online Transaction Processing}
\newacronym{kpi}{KPI}{Key Performance Indicator}
\newacronym{sla}{SLA}{Service Level Agreement}
\newacronym{roi}{ROI}{Return on Investment}
\newacronym{crm}{CRM}{Customer Relationship Management}
\newacronym{erp}{ERP}{Enterprise Resource Planning}
\newacronym{scm}{SCM}{Supply Chain Management}
\newacronym{hr}{HR}{Human Resources}
\newacronym{it}{IT}{Information Technology}
\newacronym{pfe}{PFE}{Projet de Fin d'Études}

% Définition des termes du glossaire
\newglossaryentry{microservice}{
    name=microservice,
    description={Architecture où une application est construite comme un ensemble de services faiblement couplés}
}

\newglossaryentry{conteneur}{
    name=conteneur,
    description={Unité standard de logiciel qui empaquette le code et toutes ses dépendances}
}

\newglossaryentry{pipeline}{
    name=pipeline,
    description={Séquence automatisée de processus de développement logiciel}
}

\newglossaryentry{dashboard}{
    name=dashboard,
    description={Interface utilisateur qui fournit des vues d'ensemble des métriques et indicateurs clés}
}

\newglossaryentry{scalabilite}{
    name=scalabilité,
    description={Capacité d'un système à gérer une charge croissante en ajoutant des ressources}
}

\newglossaryentry{resilience}{
    name=résilience,
    description={Capacité d'un système à continuer de fonctionner en cas de défaillance}
}

\newglossaryentry{observabilite}{
    name=observabilité,
    description={Mesure de la facilité avec laquelle l'état interne d'un système peut être inféré}
}

\newglossaryentry{agilite}{
    name=agilité,
    description={Méthodologie de développement itérative et collaborative}
}

% Usage dans le document :
% \gls{api} pour la première occurrence (forme complète)
% \gls{api} pour les occurrences suivantes (forme courte)
% \glspl{api} pour le pluriel
% \Gls{api} pour débuter une phrase
% \glsreset{api} pour réinitialiser un acronyme

% Pour imprimer le glossaire dans le document :
% \printglossary[type=\acronymtype,title=Liste des acronymes]
% \printglossary[title=Glossaire]
