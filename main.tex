\documentclass[12pt,a4paper,oneside]{report}

% Packages nécessaires
\usepackage[utf8]{inputenc}
\usepackage[T1]{fontenc}
\usepackage[english,french]{babel}
\usepackage[utf8]{inputenc}
\usepackage{pdfpages}
\usepackage[T1]{fontenc}
\usepackage[french]{babel}
\usepackage{geometry}
\usepackage{fancyhdr}
\usepackage{graphicx}
\usepackage{amsmath}
\usepackage{amsfonts}
\usepackage{amssymb}
\usepackage{hyperref}
\usepackage{url}
\usepackage{cite}
\usepackage{setspace}
\usepackage{tocloft}
\usepackage{listings}
\usepackage{xcolor}
\usepackage{float}
\usepackage{caption}
\usepackage{subcaption}
\usepackage{array}
\usepackage{longtable}
\usepackage{multirow}
\usepackage{booktabs}
\usepackage{pgfgantt}
\usepackage{tikz}
\usepackage{pdflscape}
\usepackage{titlesec}
\usetikzlibrary{positioning,shapes,arrows}

% Rendre la police plus grasse globalement
\usepackage{lmodern}
\renewcommand{\rmdefault}{lmss}  % Police sans-serif plus visible
\renewcommand{\familydefault}{\sfdefault}  % Police par défaut plus grasse

% Configuration de la page
\geometry{left=2.5cm,right=2cm,top=2cm,bottom=2cm}
\setlength{\headheight}{15pt}
\setstretch{1.3}

% Configuration des en-têtes et pieds de page
\pagestyle{fancy}
\fancyhf{}
\fancyhead[L]{\leftmark}
\fancyfoot[C]{\thepage}

% Configuration des liens hypertexte
\hypersetup{
    colorlinks=true,
    linkcolor=black,
    filecolor=magenta,      
    urlcolor=blue,
    citecolor=black
}

% Configuration pour les listings de code
\lstset{
    backgroundcolor=\color{gray!10},
    basicstyle=\ttfamily\small,
    breaklines=true,
    captionpos=b,
    commentstyle=\color{green!50!black},
    keywordstyle=\color{blue},
    stringstyle=\color{red},
    numbers=left,
    numberstyle=\tiny\color{gray},
    frame=single,
    tabsize=2
}

% Define additional languages for listings
\lstdefinelanguage{javascript}{
    keywords={break, case, catch, continue, debugger, default, delete, do, else, false, finally, for, function, if, in, instanceof, new, null, return, switch, this, throw, true, try, typeof, var, void, while, with, const, let, class, extends, export, import, super},
    morecomment=[l]{//},
    morecomment=[s]{/*}{*/},
    morestring=[b]',
    morestring=[b]",
    ndkeywords={class, export, boolean, throw, implements, import, this},
    keywordstyle=\color{blue}\bfseries,
    ndkeywordstyle=\color{darkgray}\bfseries,
    identifierstyle=\color{black},
    commentstyle=\color{purple}\ttfamily,
    stringstyle=\color{red}\ttfamily,
    sensitive=true
}

\lstdefinelanguage{yaml}{
    keywords={true, false, null},
    keywordstyle=\color{blue}\bfseries,
    basicstyle=\ttfamily,
    sensitive=false,
    comment=[l]{\#},
    morecomment=[s]{/*}{*/},
    commentstyle=\color{purple}\ttfamily,
    stringstyle=\color{red}\ttfamily,
    moredelim=[l][\color{orange}]{\&},
    moredelim=[l][\color{magenta}]{*},
    moredelim=**[il][\color{green!50!black}]{:\ }{\ },
    morestring=[b]',
    morestring=[b]"
}

\lstdefinelanguage{hcl}{
    keywords={resource, variable, output, provider, data, module, locals, terraform},
    keywordstyle=\color{blue}\bfseries,
    basicstyle=\ttfamily,
    sensitive=true,
    comment=[l]{\#},
    morecomment=[s]{/*}{*/},
    commentstyle=\color{purple}\ttfamily,
    stringstyle=\color{red}\ttfamily,
    morestring=[b]',
    morestring=[b]"
}

% Helper commands for YAML
\newcommand\YAMLcolonstyle{\color{red}\mdseries}
\newcommand\YAMLkeystyle{\color{black}\bfseries}
\newcommand\YAMLvaluestyle{\color{blue}\mdseries}
\newcommand\ProcessThreeDashes{\llap{\color{cyan}\mdseries-{-}-}}

% Définition des informations du document
\newcommand{\titre}{Mise en place d'une Infrastructure de Données Multi-sources avec Plateforme de Visualisation et Business Intelligence}
\newcommand{\auteur}{MAKROUM Ahmed}
\newcommand{\etablissement}{École Marocaine des Sciences de l'Ingénieur}
\newcommand{\organisme}{Allianz}
\newcommand{\annee}{2024-2025}
\newcommand{\encadrant}{Mr name}
\newcommand{\maitre}{Mr name}

% Formatage des titres de chapitre
\usepackage{titlesec}

% S'assurer que "Chapitre" est utilisé au lieu de "Chapter"
\addto\captionsfrench{%
  \renewcommand\chaptername{Chapitre}%
  \renewcommand\contentsname{Table des matières}%
  \renewcommand\listfigurename{Liste des figures}%
  \renewcommand\listtablename{Liste des tableaux}%
  \renewcommand\bibname{Bibliographie}%
}

% Configuration de la numérotation des sections (1, 2, 3 au lieu de 1.1, 1.2, 1.3)
\renewcommand{\thesection}{\arabic{section}}
\renewcommand{\thesubsection}{\thesection.\arabic{subsection}}
\renewcommand{\thesubsubsection}{\thesubsection.\arabic{subsubsection}}

\titleformat{\chapter}[display]
  {\normalfont\huge\bfseries\centering}
  {\chaptertitlename\ \thechapter}
  {20pt}
  {\Huge}
\titlespacing*{\chapter}{0pt}{-30pt}{40pt}

% S'assurer que les chapitres commencent sur de nouvelles pages
\let\oldchapter\chapter
\renewcommand{\chapter}{\clearpage\oldchapter}

\begin{document}

% Numérotation romaine majuscule pour les pages préliminaires
\pagenumbering{Roman}

% Page de garde
\includepdf[pages=1]{sections/PageGarde.pdf}

% Dédicace
\chapter*{Dédicace}
\addcontentsline{toc}{chapter}{Dédicace}

% Garder la numérotation romaine visible
\thispagestyle{plain}

Dédicace

\vspace{2cm}



\newpage

% Remerciements
\chapter*{Remerciements}
\addcontentsline{toc}{chapter}{Remerciements}

% Définir manuellement le mark pour les en-têtes
\markboth{Remerciements}{Remerciements}

% Garder la numérotation romaine visible
\thispagestyle{plain}

Remerciements
\vspace{2cm}



\newpage

% Résumé
\chapter*{Résumé}
\addcontentsline{toc}{chapter}{Résumé}

% Définir manuellement le mark pour les en-têtes
\markboth{Résumé}{Résumé}

% Garder la numérotation romaine visible
\thispagestyle{plain}

resume
\\

\textbf{Mots-clés~:} 

\newpage

% Résumé en arabe (mulakhass)
%\chapter*{Résumé en arabe}
\addcontentsline{toc}{chapter}{Résumé en arabe}

\thispagestyle{empty}

\begin{center}
\textbf{[RÉSUMÉ EN ARABE]}
\end{center}

\vspace{1cm}

\textbf{Contexte :} 
[Écrivez ici un bref résumé du contexte de votre projet et de l'organisme d'accueil en arabe]

\textbf{Objectif :} 
[Expliquez l'objectif principal de votre projet et ce que vous voulez accomplir]

\textbf{Méthodologie :} 
[Décrivez la méthodologie adoptée et les outils et technologies utilisés]

\textbf{Résultats :} 
[Présentez les principaux résultats obtenus et les livrables réalisés]

\textbf{Conclusion :} 
[Résumez les contributions du projet et les perspectives d'avenir]

\vspace{1cm}

\textbf{Mots-clés :} [mot-clé 1], [mot-clé 2], [mot-clé 3], [mot-clé 4], [mot-clé 5]

\newpage


% Abstract
\chapter*{Abstract}
\addcontentsline{toc}{chapter}{Abstract}

% Définir manuellement le mark pour les en-têtes
\markboth{Abstract}{Abstract}

% Garder la numérotation romaine visible
\thispagestyle{plain}

\\
\\
\\

\textbf{Keywords:} 
\newpage

% Table des matières
\tableofcontents
\markboth{Table des matières}{Table des matières}
\newpage

% Liste des figures
\listoffigures
\markboth{Liste des figures}{Liste des figures}
\newpage

% Liste des tableaux
\listoftables
\markboth{Liste des tableaux}{Liste des tableaux}
\newpage

% Liste des sigles et acronymes
\chapter*{Liste des sigles et acronymes}
\addcontentsline{toc}{chapter}{Liste des sigles et acronymes}

% Définir manuellement le mark pour les en-têtes
\markboth{Liste des sigles et acronymes}{Liste des sigles et acronymes}

% Garder la numérotation romaine visible
\thispagestyle{plain}

\begin{tabular}{ll}
\textbf{PoC} & Proof of Concept \\
\textbf{PGSQL} & PostgresSQL: Système de gestion de base de données relationnelle \\
\textbf{Spark} & Apache Spark (moteur de traitement distribué de données) \\
\textbf{SQL} & Structured Query Language \\
\end{tabular}

\newpage


% Passage à la numérotation arabe pour le contenu principal
\pagenumbering{arabic}

% Réinitialiser les en-têtes et pieds de page pour le contenu principal
\pagestyle{fancy}
\fancyhf{}
\fancyhead[L]{\leftmark}
\fancyfoot[C]{\thepage}

% Introduction générale
\chapter*{Introduction Générale}
\addcontentsline{toc}{chapter}{Introduction Générale}
\thispagestyle{plain}
\pagestyle{plain}
\markboth{}{}

Introduction Générale


\newpage


% Restaurer le style de page normal pour les chapitres
\clearpage
\pagestyle{fancy}
\markboth{}{}

% Chapitre 1: Contexte général du projet
\chapter{Contexte général du Projet}

\section*{Introduction}
\addcontentsline{toc}{section}{Introduction}



\subsection{Présentation générale}

\subsection{Organisation}


\subsection{Principes fondamentaux d’Allianz}




\section{Cadrage du projet}

\subsection{Étude de l'existant}

\subsection{Problématique}

\subsection{Solution proposée}


\subsection{Objectifs du projet}

\section{Concepts fondamentaux : ETL, Big Data et l’Architecture Décisionnelle Moderne}


\section{Conduite du projet}


\subsection{Diagramme de Gantt}


\subsection{Diagramme de cas d'utilisation}


\section{Conclusion}

\newpage

% Chapitre 2: Analyse et modélisation
\chapter{Analyse et Modélisation}

\section*{Introduction}

\section{Cahier des charges}


\section{Modélisation du Data Warehouse}


\section{Qualité des données et gouvernance}


\section{Conclusion}


\newpage

% Chapitre 3: Étude technique
\chapter{Étude technique}

\section*{Introduction}
\addcontentsline{toc}{section}{Introduction}



\section{Architecture finale}


\section{Conclusion}


\newpage

% Chapitre 4: Réalisation et déploiement
\chapter{Réalisation et déploiement}

\section*{Introduction}

\section{Création du PoC}


\subsection{Configuration de l'environnement de développement}
\subsection{Développement de la solution}

\section{DevOps et déploiement}


\subsection{Orchestration avec Docker Compose}


\subsection{Tests et validation de la solution}


\section{Conclusion}



% Conclusion générale
\newpage
\chapter*{Conclusion Générale}
\addcontentsline{toc}{chapter}{Conclusion Générale}
\markboth{Conclusion Générale}{Conclusion Générale}

\section*{Synthèse du projet}


\section*{Difficultés rencontrées et résolutions}

\subsection*{Techniques}


\subsection*{Organisationnelles}



\section*{Perspectives d’évolution}

\subsection*{Court terme}


\subsection*{Moyen terme}


\subsection*{Long terme}



\section*{Retour d’expérience}



\newpage




% Bibliographie
\bibliographystyle{plain}
\bibliography{references}

% Webographie
\chapter*{Webographie}
\addcontentsline{toc}{chapter}{Webographie}
\markboth{Webographie}{Webographie}

\section*{Sites web consultés}
\begin{enumerate}
    \item \textbf{PostgreSQL Documentation} \\
    \url{https://www.postgresql.org/docs/} \\
    Consulté le : \textbf{8 mars 2025} \\
    Description : Documentation officielle et complète de PostgreSQL.

    \item \textbf{Docker Documentation} \\
    \url{https://docs.docker.com/} \\
    Consulté le : \textbf{19 mars 2025} \\
    Description : Guide officiel pour la containerisation, les images et les déploiements.

    \item \textbf{Kubernetes Documentation} \\
    \url{https://kubernetes.io/docs/} \\
    Consulté le : \textbf{5 mars 2025} \\
    Description : Documentation et meilleures pratiques pour l’orchestration de conteneurs.

    \item \textbf{Apache NiFi Documentation} \\
    \url{https://nifi.apache.org/documentation/} \\
    Consulté le : \textbf{13 mars 2025} \\
    Description : Guides et référence de NiFi, outil d’orchestration de flux de données.

    \item \textbf{GitHub Actions Documentation} \\
    \url{https://docs.github.com/en/actions} \\
    Consulté le : \textbf{22 mars 2025} \\
    Description : Documentation officielle sur l’automatisation CI/CD via GitHub Actions.

    \item \textbf{DevOps Best Practices (Azure DevOps)} \\
    \url{https://docs.microsoft.com/azure/devops/} \\
    Consulté le : \textbf{10 mars 2025} \\
    Description : Guide Microsoft pour les pratiques DevOps incluant CI/CD.

    \item \textbf{ChatGPT (OpenAI)} \\
    \url{https://chatgpt.com/} \\
    Consulté le : \textbf{3 mars 2025} \\
    Description : Assistant IA conversationnel basé sur GPT, pour la rédaction, la recherche et le codage.

    \item \textbf{Claude (Anthropic)} \\
    \url{https://claude.ai/} \\
    Consulté le : \textbf{3 mars 2025} \\
    Description : Assistant IA alternatif axé sur la sécurité, la précision et la productivité.

    \item \textbf{GitHub Copilot} \\
    \url{https://github.com/features/copilot} \\
    Consulté le : \textbf{3 mars 2025} \\
    Description : Assistant IA de programmation pour l’autocomplétion et la génération de code.

    \item \textbf{Metabase Documentation} \\
    \url{https://www.metabase.com/docs/} \\
    Consulté le : \textbf{12 mars 2025} \\
    Description : Documentation officielle de Metabase pour la visualisation et l’analyse de données.

    \item \textbf{Apache Spark Documentation} \\
    \url{https://spark.apache.org/docs/latest/} \\
    Consulté le : \textbf{15 mars 2025} \\
    Description : Référence technique pour le traitement distribué et l’analytique big data.

    \item \textbf{OWASP Top Ten} \\
    \url{https://owasp.org/www-project-top-ten/} \\
    Consulté le : \textbf{21 mars 2025} \\
    Description : Guide des risques majeurs de sécurité applicative, utile pour la sécurisation des APIs et des plateformes BI.

    \item \textbf{Data Engineering Podcast} \\
    \url{https://www.dataengineeringpodcast.com/} \\
    Consulté le : \textbf{18 mars 2025} \\
    Description : Podcast sur les architectures, outils et retours d’expérience en data engineering.

    \item \textbf{Allianz Group} \\
    \url{https://www.allianz.com/en.html} \\
    Consulté le : \textbf{2 mars 2025} \\
    Description : Site institutionnel du groupe Allianz, pour la compréhension du contexte métier et des enjeux sectoriels.
\end{enumerate}

\section*{Tutoriels et articles techniques}
\begin{enumerate}
    \item \textbf{Building Scalable Data Pipelines} \\
    \url{https://medium.com/data-engineering} \\
    Consulté le : \textbf{14 mars 2025} \\
    Description : Articles techniques sur la conception de pipelines de données évolutifs.

    \item \textbf{Start Data Engineering (blog)} \\
    \url{https://www.startdataengineering.com/} \\
    Consulté le : \textbf{25 mars 2025} \\
    Description : Blog spécialisé sur les pratiques, outils et architectures en data engineering.

    \item \textbf{Kubernetes Best Practices} \\
    \url{https://cloud.google.com/kubernetes-engine/docs/best-practices} \\
    Consulté le : \textbf{7 mars 2025} \\
    Description : Bonnes pratiques pour la mise en œuvre de Kubernetes sur Google Cloud.

    \item \textbf{Database Design Patterns (PostgreSQL)} \\
    \url{https://www.postgresql.org/docs/current/ddl.html} \\
    Consulté le : \textbf{17 mars 2025} \\
    Description : Patterns et bonnes pratiques pour la conception de schémas relationnels.

    \item \textbf{API Design Guidelines (REST)} \\
    \url{https://restfulapi.net/} \\
    Consulté le : \textbf{11 mars 2025} \\
    Description : Directives pour la conception, la normalisation et la sécurisation des APIs REST.

    \item \textbf{Modern Data Stack (Fivetran Blog)} \\
    \url{https://fivetran.com/blog/modern-data-stack} \\
    Consulté le : \textbf{16 mars 2025} \\
    Description : Analyse des tendances et outils de la data stack moderne.

    \item \textbf{DataOps Manifesto} \\
    \url{https://www.dataopsmanifesto.org/} \\
    Consulté le : \textbf{20 mars 2025} \\
    Description : Principes et bonnes pratiques pour l’industrialisation des pipelines de données.

    \item \textbf{YouTube – Apache NiFi Tutorials (DataCouch)} \\
    \url{https://www.youtube.com/playlist?list=PLf0swTFhTI8q18p6U6QvFqQnQKp1K0QdC} \\
    Consulté le : \textbf{9 mars 2025} \\
    Description : Tutoriels vidéo pour la prise en main et l’automatisation de flux avec NiFi.

    \item \textbf{Towards Data Science (Medium)} \\
    \url{https://towardsdatascience.com/} \\
    Consulté le : \textbf{23 mars 2025} \\
    Description : Articles de vulgarisation et d’approfondissement sur la data science, l’IA et l’ingénierie des données.

    \item \textbf{Awesome Data Engineering (GitHub)} \\
    \url{https://github.com/igorbarinov/awesome-data-engineering} \\
    Consulté le : \textbf{27 mars 2025} \\
    Description : Liste collaborative de ressources, outils et lectures recommandées pour l’ingénierie des données.

    \item \textbf{Le Monde Informatique – Dossiers Data} \\
    \url{https://www.lemondeinformatique.fr/dossiers/data/} \\
    Consulté le : \textbf{28 février 2025} \\
    Description : Dossiers et actualités sur la transformation digitale, la gouvernance et la valorisation des données en entreprise.

    \item \textbf{Qwen (Alibaba Cloud)} \\
    \url{https://qwen.alibaba.com/} \\
    Consulté le : \textbf{4 mars 2025} \\
    Description : Modèle d’IA générative open source développé par Alibaba Cloud, utilisé pour la génération de texte et l’assistance à la rédaction technique.

    \item \textbf{Gemini (Google AI)} \\
    \url{https://deepmind.google/technologies/gemini/} \\
    Consulté le : \textbf{8 mars 2025} \\
    Description : Modèle d’IA multimodal de Google, utilisé pour la recherche, la génération de code et l’analyse de données complexes.

    \item \textbf{Google Cloud Copilot} \\
    \url{https://cloud.google.com/ai/copilot} \\
    Consulté le : \textbf{12 mars 2025} \\
    Description : Assistant IA de Google Cloud pour l’aide à la programmation, la génération de code et l’automatisation des workflows cloud.
\end{enumerate}


% Effacer les en-têtes à la fin du document
\clearpage
\pagestyle{empty}
\markboth{}{}

\end{document}
