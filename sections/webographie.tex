\chapter*{Webographie}
\addcontentsline{toc}{chapter}{Webographie}
\markboth{Webographie}{Webographie}

\section*{Sites web consultés}
\begin{enumerate}
    \item \textbf{PostgreSQL Documentation} \\
    \url{https://www.postgresql.org/docs/} \\
    Consulté le : \textbf{8 mars 2025} \\
    Description : Documentation officielle et complète de PostgreSQL.

    \item \textbf{Docker Documentation} \\
    \url{https://docs.docker.com/} \\
    Consulté le : \textbf{19 mars 2025} \\
    Description : Guide officiel pour la containerisation, les images et les déploiements.

    \item \textbf{Kubernetes Documentation} \\
    \url{https://kubernetes.io/docs/} \\
    Consulté le : \textbf{5 mars 2025} \\
    Description : Documentation et meilleures pratiques pour l’orchestration de conteneurs.

    \item \textbf{Apache NiFi Documentation} \\
    \url{https://nifi.apache.org/documentation/} \\
    Consulté le : \textbf{13 mars 2025} \\
    Description : Guides et référence de NiFi, outil d’orchestration de flux de données.

    \item \textbf{GitHub Actions Documentation} \\
    \url{https://docs.github.com/en/actions} \\
    Consulté le : \textbf{22 mars 2025} \\
    Description : Documentation officielle sur l’automatisation CI/CD via GitHub Actions.

    \item \textbf{DevOps Best Practices (Azure DevOps)} \\
    \url{https://docs.microsoft.com/azure/devops/} \\
    Consulté le : \textbf{10 mars 2025} \\
    Description : Guide Microsoft pour les pratiques DevOps incluant CI/CD.

    \item \textbf{ChatGPT (OpenAI)} \\
    \url{https://chatgpt.com/} \\
    Consulté le : \textbf{3 mars 2025} \\
    Description : Assistant IA conversationnel basé sur GPT, pour la rédaction, la recherche et le codage.

    \item \textbf{Claude (Anthropic)} \\
    \url{https://claude.ai/} \\
    Consulté le : \textbf{3 mars 2025} \\
    Description : Assistant IA alternatif axé sur la sécurité, la précision et la productivité.

    \item \textbf{GitHub Copilot} \\
    \url{https://github.com/features/copilot} \\
    Consulté le : \textbf{3 mars 2025} \\
    Description : Assistant IA de programmation pour l’autocomplétion et la génération de code.

    \item \textbf{Metabase Documentation} \\
    \url{https://www.metabase.com/docs/} \\
    Consulté le : \textbf{12 mars 2025} \\
    Description : Documentation officielle de Metabase pour la visualisation et l’analyse de données.

    \item \textbf{Apache Spark Documentation} \\
    \url{https://spark.apache.org/docs/latest/} \\
    Consulté le : \textbf{15 mars 2025} \\
    Description : Référence technique pour le traitement distribué et l’analytique big data.

    \item \textbf{OWASP Top Ten} \\
    \url{https://owasp.org/www-project-top-ten/} \\
    Consulté le : \textbf{21 mars 2025} \\
    Description : Guide des risques majeurs de sécurité applicative, utile pour la sécurisation des APIs et des plateformes BI.

    \item \textbf{Data Engineering Podcast} \\
    \url{https://www.dataengineeringpodcast.com/} \\
    Consulté le : \textbf{18 mars 2025} \\
    Description : Podcast sur les architectures, outils et retours d’expérience en data engineering.

    \item \textbf{Allianz Group} \\
    \url{https://www.allianz.com/en.html} \\
    Consulté le : \textbf{2 mars 2025} \\
    Description : Site institutionnel du groupe Allianz, pour la compréhension du contexte métier et des enjeux sectoriels.
\end{enumerate}

\section*{Tutoriels et articles techniques}
\begin{enumerate}
    \item \textbf{Building Scalable Data Pipelines} \\
    \url{https://medium.com/data-engineering} \\
    Consulté le : \textbf{14 mars 2025} \\
    Description : Articles techniques sur la conception de pipelines de données évolutifs.

    \item \textbf{Start Data Engineering (blog)} \\
    \url{https://www.startdataengineering.com/} \\
    Consulté le : \textbf{25 mars 2025} \\
    Description : Blog spécialisé sur les pratiques, outils et architectures en data engineering.

    \item \textbf{Kubernetes Best Practices} \\
    \url{https://cloud.google.com/kubernetes-engine/docs/best-practices} \\
    Consulté le : \textbf{7 mars 2025} \\
    Description : Bonnes pratiques pour la mise en œuvre de Kubernetes sur Google Cloud.

    \item \textbf{Database Design Patterns (PostgreSQL)} \\
    \url{https://www.postgresql.org/docs/current/ddl.html} \\
    Consulté le : \textbf{17 mars 2025} \\
    Description : Patterns et bonnes pratiques pour la conception de schémas relationnels.

    \item \textbf{API Design Guidelines (REST)} \\
    \url{https://restfulapi.net/} \\
    Consulté le : \textbf{11 mars 2025} \\
    Description : Directives pour la conception, la normalisation et la sécurisation des APIs REST.

    \item \textbf{Modern Data Stack (Fivetran Blog)} \\
    \url{https://fivetran.com/blog/modern-data-stack} \\
    Consulté le : \textbf{16 mars 2025} \\
    Description : Analyse des tendances et outils de la data stack moderne.

    \item \textbf{DataOps Manifesto} \\
    \url{https://www.dataopsmanifesto.org/} \\
    Consulté le : \textbf{20 mars 2025} \\
    Description : Principes et bonnes pratiques pour l’industrialisation des pipelines de données.

    \item \textbf{YouTube – Apache NiFi Tutorials (DataCouch)} \\
    \url{https://www.youtube.com/playlist?list=PLf0swTFhTI8q18p6U6QvFqQnQKp1K0QdC} \\
    Consulté le : \textbf{9 mars 2025} \\
    Description : Tutoriels vidéo pour la prise en main et l’automatisation de flux avec NiFi.

    \item \textbf{Towards Data Science (Medium)} \\
    \url{https://towardsdatascience.com/} \\
    Consulté le : \textbf{23 mars 2025} \\
    Description : Articles de vulgarisation et d’approfondissement sur la data science, l’IA et l’ingénierie des données.

    \item \textbf{Awesome Data Engineering (GitHub)} \\
    \url{https://github.com/igorbarinov/awesome-data-engineering} \\
    Consulté le : \textbf{27 mars 2025} \\
    Description : Liste collaborative de ressources, outils et lectures recommandées pour l’ingénierie des données.

    \item \textbf{Le Monde Informatique – Dossiers Data} \\
    \url{https://www.lemondeinformatique.fr/dossiers/data/} \\
    Consulté le : \textbf{28 février 2025} \\
    Description : Dossiers et actualités sur la transformation digitale, la gouvernance et la valorisation des données en entreprise.

    \item \textbf{Qwen (Alibaba Cloud)} \\
    \url{https://qwen.alibaba.com/} \\
    Consulté le : \textbf{4 mars 2025} \\
    Description : Modèle d’IA générative open source développé par Alibaba Cloud, utilisé pour la génération de texte et l’assistance à la rédaction technique.

    \item \textbf{Gemini (Google AI)} \\
    \url{https://deepmind.google/technologies/gemini/} \\
    Consulté le : \textbf{8 mars 2025} \\
    Description : Modèle d’IA multimodal de Google, utilisé pour la recherche, la génération de code et l’analyse de données complexes.

    \item \textbf{Google Cloud Copilot} \\
    \url{https://cloud.google.com/ai/copilot} \\
    Consulté le : \textbf{12 mars 2025} \\
    Description : Assistant IA de Google Cloud pour l’aide à la programmation, la génération de code et l’automatisation des workflows cloud.
\end{enumerate}
